\documentclass[fleqn]{beamer}\usepackage[]{graphicx}\usepackage[]{color}
%% maxwidth is the original width if it is less than linewidth
%% otherwise use linewidth (to make sure the graphics do not exceed the margin)
\makeatletter
\def\maxwidth{ %
  \ifdim\Gin@nat@width>\linewidth
    \linewidth
  \else
    \Gin@nat@width
  \fi
}
\makeatother

\definecolor{fgcolor}{rgb}{0.345, 0.345, 0.345}
\newcommand{\hlnum}[1]{\textcolor[rgb]{0.686,0.059,0.569}{#1}}%
\newcommand{\hlstr}[1]{\textcolor[rgb]{0.192,0.494,0.8}{#1}}%
\newcommand{\hlcom}[1]{\textcolor[rgb]{0.678,0.584,0.686}{\textit{#1}}}%
\newcommand{\hlopt}[1]{\textcolor[rgb]{0,0,0}{#1}}%
\newcommand{\hlstd}[1]{\textcolor[rgb]{0.345,0.345,0.345}{#1}}%
\newcommand{\hlkwa}[1]{\textcolor[rgb]{0.161,0.373,0.58}{\textbf{#1}}}%
\newcommand{\hlkwb}[1]{\textcolor[rgb]{0.69,0.353,0.396}{#1}}%
\newcommand{\hlkwc}[1]{\textcolor[rgb]{0.333,0.667,0.333}{#1}}%
\newcommand{\hlkwd}[1]{\textcolor[rgb]{0.737,0.353,0.396}{\textbf{#1}}}%
\let\hlipl\hlkwb

\usepackage{framed}
\makeatletter
\newenvironment{kframe}{%
 \def\at@end@of@kframe{}%
 \ifinner\ifhmode%
  \def\at@end@of@kframe{\end{minipage}}%
  \begin{minipage}{\columnwidth}%
 \fi\fi%
 \def\FrameCommand##1{\hskip\@totalleftmargin \hskip-\fboxsep
 \colorbox{shadecolor}{##1}\hskip-\fboxsep
     % There is no \\@totalrightmargin, so:
     \hskip-\linewidth \hskip-\@totalleftmargin \hskip\columnwidth}%
 \MakeFramed {\advance\hsize-\width
   \@totalleftmargin\z@ \linewidth\hsize
   \@setminipage}}%
 {\par\unskip\endMakeFramed%
 \at@end@of@kframe}
\makeatother

\definecolor{shadecolor}{rgb}{.97, .97, .97}
\definecolor{messagecolor}{rgb}{0, 0, 0}
\definecolor{warningcolor}{rgb}{1, 0, 1}
\definecolor{errorcolor}{rgb}{1, 0, 0}
\newenvironment{knitrout}{}{} % an empty environment to be redefined in TeX

\usepackage{alltt}
\usepackage[utf8]{inputenc}
\usepackage[brazilian]{babel}
\usepackage{amsmath}
\usepackage{enumerate}
\usepackage[]{graphicx}
\usepackage[]{color}
\usepackage{pgfplots}
\usetheme{Berkeley}

\pgfplotsset{compat=1.5}
\IfFileExists{upquote.sty}{\usepackage{upquote}}{}
\begin{document}
	\section{Questão 1}
	\section{Questão 2}
	\section{Questão 3}
	\section{Questão 4}
	\section{Questão 5}
	\section{Questão 6}
		\begin{frame}{Questão 6}
			a)  Sejam $\overline{\overline X} = 34.32$ e $\overline{\overline R} = 5.65$.\\
			Para amostras de tamanho \textbf{5}, $A_{2} = 0.577$.
			Limites para $\overline{X}$ são:
			$$\overline{X} \pm A_{2}\sigma = 34.32 \pm (0.577)(5.65) = 34.32 \pm 3.26$$
			ou
			$$LSC = 37.58\hspace{1cm}LIC = 31.06$$
			Para o gráfico \textbf{R}, os limites de controle são:
			$$LSC = D_{4}\overline{R} = (2.115)(5.65) = 11.95$$
			$$LIC = D_{3}\overline{R} = 0,$$
			sendo $D_{4}$ e $D_{3}$ tabelados.
		\end{frame}
		\begin{frame}{Questão 6}
			\centering
			\begin{tikzpicture}
				\begin{axis}[
					title=Gráfico para $\overline{X}$,
					xlabel={Amostra},
					ylabel={$\overline X$},
					xtick align=outside,
					ytick align=outside,
					xtick pos=left,
					ytick pos=left,
					minor x tick num=4,
					minor y tick num=5,
					xmin=0
					]
					\addplot table[col sep=semicolon] {q6X.csv};
				\end{axis}
			\end{tikzpicture}
		\end{frame}
		\begin{frame}{Questão 6}
			\centering
			\begin{tikzpicture}
				\begin{axis}[
				title=Gráfico para $R$,
				xlabel={Amostra},
				ylabel={R},
				xtick align=outside,
				ytick align=outside,
				xtick pos=left,
				ytick pos=left,
				xmin=0
				]
				\addplot [red, mark=*] table[col sep=semicolon] {q6R.csv};
				\end{axis}
			\end{tikzpicture}
		\end{frame}
		\begin{frame}{Questão 6}
			b)  $RCP = \frac{LSE - LIE}{6\delta}$, onde:$$\delta = \frac{r}{d_{2}(tabelado)} = \frac{5}{2\times32.6} = 0.000215$$\\
			$RCP = \frac{0.5045 - 0.5025}{6(0.000215)} = 1.55$\\
			
			$P_{1} = \big(\frac{1}{C_{o}}\big)\times100 = 64.51$
		\end{frame}
	\section{Questão 7}
	\section{Questão 8}
		\begin{frame}{Questão 8}
			a)  Sejam $TMF = 0.002$, $t = 1000h$ e $r(t) = \exp(-\lambda t)$.
			$$r(1000) = \exp(-0.002\times1000) = 0.135$$
			Assim:
			\begin{align*}
				R(t) &= \sum_{l = k}^{n} \binom{n}{k} [r(t)] ^{l} [1-r(t)] ^{n-l}\\
					&= \sum_{l = 2}^{5} \binom{5}{2} [0.135] ^{2} [1-0.135] ^{5-2}\\
				R(t) &=  0.417
			\end{align*}
		\end{frame}
		\begin{frame}{Questão 8}
			b)\begin{align*}
					R(t) &= 1 - [1 - \exp(-\lambda t)]^{n}\\
						&= 1 - [1 - \exp(-0.002\times1000)]^{5}\\
					R(t) &= 0.516
			\end{align*}
		\end{frame}
	\section{Questão 9}

\end{document}

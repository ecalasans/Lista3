\documentclass[fleqn]{beamer}\usepackage[]{graphicx}\usepackage[]{color}
%% maxwidth is the original width if it is less than linewidth
%% otherwise use linewidth (to make sure the graphics do not exceed the margin)
\makeatletter
\def\maxwidth{ %
  \ifdim\Gin@nat@width>\linewidth
    \linewidth
  \else
    \Gin@nat@width
  \fi
}
\makeatother

\definecolor{fgcolor}{rgb}{0.345, 0.345, 0.345}
\newcommand{\hlnum}[1]{\textcolor[rgb]{0.686,0.059,0.569}{#1}}%
\newcommand{\hlstr}[1]{\textcolor[rgb]{0.192,0.494,0.8}{#1}}%
\newcommand{\hlcom}[1]{\textcolor[rgb]{0.678,0.584,0.686}{\textit{#1}}}%
\newcommand{\hlopt}[1]{\textcolor[rgb]{0,0,0}{#1}}%
\newcommand{\hlstd}[1]{\textcolor[rgb]{0.345,0.345,0.345}{#1}}%
\newcommand{\hlkwa}[1]{\textcolor[rgb]{0.161,0.373,0.58}{\textbf{#1}}}%
\newcommand{\hlkwb}[1]{\textcolor[rgb]{0.69,0.353,0.396}{#1}}%
\newcommand{\hlkwc}[1]{\textcolor[rgb]{0.333,0.667,0.333}{#1}}%
\newcommand{\hlkwd}[1]{\textcolor[rgb]{0.737,0.353,0.396}{\textbf{#1}}}%
\let\hlipl\hlkwb

\usepackage{framed}
\makeatletter
\newenvironment{kframe}{%
 \def\at@end@of@kframe{}%
 \ifinner\ifhmode%
  \def\at@end@of@kframe{\end{minipage}}%
  \begin{minipage}{\columnwidth}%
 \fi\fi%
 \def\FrameCommand##1{\hskip\@totalleftmargin \hskip-\fboxsep
 \colorbox{shadecolor}{##1}\hskip-\fboxsep
     % There is no \\@totalrightmargin, so:
     \hskip-\linewidth \hskip-\@totalleftmargin \hskip\columnwidth}%
 \MakeFramed {\advance\hsize-\width
   \@totalleftmargin\z@ \linewidth\hsize
   \@setminipage}}%
 {\par\unskip\endMakeFramed%
 \at@end@of@kframe}
\makeatother

\definecolor{shadecolor}{rgb}{.97, .97, .97}
\definecolor{messagecolor}{rgb}{0, 0, 0}
\definecolor{warningcolor}{rgb}{1, 0, 1}
\definecolor{errorcolor}{rgb}{1, 0, 0}
\newenvironment{knitrout}{}{} % an empty environment to be redefined in TeX

\usepackage{alltt}
\usepackage[utf8]{inputenc}
\usepackage[brazilian]{babel}
\usepackage{amsmath}
\usepackage{enumerate}
\usepackage[]{graphicx}
\usepackage[]{color}
\usepackage{pgfplots}
\usepackage{systeme}
\usetheme{Berkeley}

\pgfplotsset{compat=1.5}
\IfFileExists{upquote.sty}{\usepackage{upquote}}{}
\begin{document}
	\title{Lista 2}
	\subtitle{Tópicos Especiais em Engenharia de Computação}
	\author{Eric Calasans de Barros \and  Fagner Freire de Oliveira}
	
	\begin{frame}[plain]
		\maketitle
	\end{frame}
	\section{Questão 1}
	\section{Questão 2}
	\section{Questão 3}
	\section{Questão 4}
	\section{Questão 5}
		\begin{frame}{Questão 5}
			a)  Seja $\hat{y}(x) = a_{1}x + a_{0}$.  Os coeficientes são dados por:
			$$a_{1} = \frac{n\sum (xy) - \sum x \sum y}{n\sum x^{2} - (\sum x)^2}$$
			$$a_{0} = \bar{y} - a_{1}\bar{x}$$
			Para n = 13 e com os dados da tabela temos:
			$$\sum x = 260 \hspace{1cm} \sum xy = 3580$$
			$$\sum y = 214 \hspace{1cm} \sum x^{2} = 6808$$
			$$\bar{y} = 16.46 \hspace{1cm} \bar{x} = 20$$
		\end{frame}
		\begin{frame}{Questão 5}
			$$a_{1} = \frac{13\times3380 - 260\times214}{13\times6808 - 260^{2}} = -0.56$$
			$$a_{0} = 16.46 - (-0.56\times20) = 27.66$$
			$$\sigma^{2} = 63.48 \hspace{1cm} \sigma = 7.98$$
		\end{frame}
		\begin{frame}{Questão 5}
			c)  Para um modelo $\hat{y} = a_{0} + a_{1}x_{1} + a_{2}x_{2}$, temos as seguintes equações:
			\begin{align*}
				na_{0} + a_{1}\sum_{i=1}^{n} x_{i1} + a_{2}\sum_{i=1}^{n} x_{i2} &= \sum_{i=1}^{n} y_{i1}\\
				a_{0}\sum_{i=1}^{n} x_{i1} + a_{1}\sum_{i=1}^{n} x_{i1}^{2} + a_{2}\sum_{i=1}^{n} x_{i1} x_{i2} &= \sum_{i=1}^{n} x_{i1} y_{i1}\\
				a_{0}\sum_{i=1}^{n} x_{i2} + a_{1}\sum_{i=1}^{n} x_{i1} x_{i2} + a_{2}\sum_{i=1}^{n} x_{i2}^{i2} &= \sum_{i=1}^{n} x_{i2} y_{i}   
			\end{align*}
		\end{frame}
		\begin{frame}{Questão 5}
			Calculando n e os somatórios:
			$$n = 8\hspace{1cm} \sum y_{i} = 227.7 \hspace{1cm} \sum x_{i1} = 20$$
			$$\sum x_{i2} = 12 \hspace{1cm} \sum x_{i1}^2 = 60 \hspace{1cm} \sum x_{i1} x_{i2} = 30$$
			$$\sum x_{i1} y_{i} = 661 \hspace{1cm} \sum x_{i2} y_{i} = 331.2$$
		\end{frame}
		\begin{frame}{Questão 5}
			\[
				\systeme*{8a_{0} + 20a_{1} + 12a_{2} = 227.7,20a_{0} + 60a_{1} + 30a_{2} = 661,12a_{0} + 30a_{1} + 20a_{2} = 331.2}
			\]
			Resolvendo o sistema acima temos: $a_{0} = 13.29$, $a_{1} = 9.18$ e $a_{2} = -5.18$.
			Asssim:
			\[ \hat{y} = 13.29 + 9.18x_{1} - 5.18x_{2}\]
		\end{frame}
		\begin{frame}{Questão 5}
			Para $x_{1}$:  $\bar{x}_{1} = 2.5$, $\sigma^{2} = 1.25$ e $\sigma = 1.118$.\\
			Para $x_{2}$:  $\bar{x}_{2} = 1.5$, $\sigma^{2} = 0.25$ e $\sigma = 0.5$.\\
			Para $y$:  $\bar{y} = 28.46$, $\sigma^{2} = 112.39$ e $\sigma = 10.6$.
			$$S_{t} = 899.14 \hspace{2cm} S_{r} = 3.78$$
			$$T = \sqrt{\frac{S_{t} - S{r}}{S_{t}}} = \sqrt{\frac{899.14 - 3.78}{3.78}} = 15.39$$			
		\end{frame}
	\section{Questão 6}
		\begin{frame}{Questão 6}
			a)  Sejam $\overline{\overline X} = 34.32$ e $\overline{\overline R} = 5.65$.\\
			Para amostras de tamanho \textbf{5}, $A_{2} = 0.577$.
			Limites para $\overline{X}$ são:
			$$\overline{X} \pm A_{2}\sigma = 34.32 \pm (0.577)(5.65) = 34.32 \pm 3.26$$
			ou
			$$LSC = 37.58\hspace{1cm}LIC = 31.06$$
			Para o gráfico \textbf{R}, os limites de controle são:
			$$LSC = D_{4}\overline{R} = (2.115)(5.65) = 11.95$$
			$$LIC = D_{3}\overline{R} = 0,$$
			sendo $D_{4}$ e $D_{3}$ tabelados.
		\end{frame}
		\begin{frame}{Questão 6}
			\centering
			\begin{tikzpicture}
				\begin{axis}[
					title=Gráfico para $\overline{X}$,
					xlabel={Amostra},
					ylabel={$\overline X$},
					xtick align=outside,
					ytick align=outside,
					xtick pos=left,
					ytick pos=left,
					minor x tick num=4,
					minor y tick num=5,
					xmin=0
					]
					\addplot table[col sep=semicolon] {q6X.csv};
				\end{axis}
			\end{tikzpicture}
		\end{frame}
		\begin{frame}{Questão 6}
			\centering
			\begin{tikzpicture}
				\begin{axis}[
				title=Gráfico para $R$,
				xlabel={Amostra},
				ylabel={R},
				xtick align=outside,
				ytick align=outside,
				xtick pos=left,
				ytick pos=left,
				xmin=0
				]
				\addplot [red, mark=*] table[col sep=semicolon] {q6R.csv};
				\end{axis}
			\end{tikzpicture}
		\end{frame}
		\begin{frame}{Questão 6}
			b)  $RCP = \frac{LSE - LIE}{6\delta}$, onde:$$\delta = \frac{r}{d_{2}(tabelado)} = \frac{5}{2\times32.6} = 0.000215$$\\
			$RCP = \frac{0.5045 - 0.5025}{6(0.000215)} = 1.55$\\
			
			c)  $P_{c} = \big(\frac{1}{C_{p}}\big)\times100 = 64.51$
		\end{frame}
	\section{Questão 7}
		\begin{frame}{Questão 7}
			Sejam:
			$$\bar{c} = \frac{c_{1} + c_{2} + \dots + c_{k}}{k}$$
			$$LSC = \bar{c} + 3\sqrt{\bar{c}} \hspace{1cm} LC = \bar{c} \hspace{1cm} LIC = \bar{c} - 3\sqrt{\bar{c}}$$
			Para $\bar{c} = 9.7$:
			$$LSC = 9.7 + 3\sqrt{9.7} = 19.04 \hspace{1cm} LIC = 9.7 - 3\sqrt{9.7} = 0.35$$
			A 4ª e a 24ª amostras estão fora dos limites de controle.  Revendo os limites e retirando essas amostras temos:
		\end{frame}
		\begin{frame}{Questão 7}
			\begin{align*}
				&\bar{c} = 9.4\\
				&LSC = 9.4 + 3\sqrt{9.4} = 18.6\\
				&LIC = 9.4 - 3\sqrt{9.4} = 0.2 
			\end{align*}
			Retirando as observações 4 e 24 os dados encontram-se agora dentro dos limites de controle.
		\end{frame}
	\section{Questão 8}
		\begin{frame}{Questão 8}
			a)  Sejam $TMF = 0.002$, $t = 1000h$ e $r(t) = \exp(-\lambda t)$.
			$$r(1000) = \exp(-0.002\times1000) = 0.135$$
			Assim:
			\begin{align*}
				R(t) &= \sum_{l = k}^{n} \binom{n}{k} [r(t)] ^{l} [1-r(t)] ^{n-l}\\
					&= \sum_{l = 2}^{5} \binom{5}{2} [0.135] ^{2} [1-0.135] ^{5-2}\\
				R(t) &=  0.417
			\end{align*}
		\end{frame}
		\begin{frame}{Questão 8}
			b)\begin{align*}
					R(t) &= 1 - [1 - \exp(-\lambda t)]^{n}\\
						&= 1 - [1 - \exp(-0.002\times1000)]^{5}\\
					R(t) &= 0.516
			\end{align*}
		\end{frame}
	\section{Questão 9}

\end{document}
